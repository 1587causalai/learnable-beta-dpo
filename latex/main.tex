\documentclass[conference]{IEEEtran}
\usepackage{cite}
\usepackage{amsmath,amssymb,amsfonts}
\usepackage{graphicx}
\usepackage{xcolor}
\usepackage{hyperref}
\usepackage{booktabs}
\usepackage{CJKutf8}

\begin{document}
\begin{CJK}{UTF8}{gbsn}

\title{可学习Beta值的DPO算法研究:自适应探索-利用平衡的新方法}

\author{\IEEEauthorblockN{作者名}
\IEEEauthorblockA{机构\\
邮箱}}

\maketitle

\begin{abstract}
本文提出了一种创新的Direct Preference Optimization (DPO)算法变体——Learnable Beta DPO,通过引入动态可学习的β参数来实现对探索-利用平衡的自适应控制。传统DPO算法使用固定的β超参数来平衡参考策略和偏好学习,这限制了其在复杂多变场景下的优化潜力。我们设计了一个与策略模型紧密耦合的BetaHead网络,能够根据输入上下文动态调整β值,从而在模型熟悉的领域保持保守学习策略,在不熟悉的领域加大探索力度。实验结果表明,Learnable Beta DPO相比固定β的标准DPO,在性能、泛化能力和样本效率方面均有显著提升。
\end{abstract}

\begin{IEEEkeywords}
Direct Preference Optimization, 大语言模型, 人类偏好对齐, 自适应学习, 信息融合
\end{IEEEkeywords}

\section{引言}
\subsection{研究背景与意义}
Direct Preference Optimization (DPO) \cite{rafailov2023direct} 作为一种直接优化语言模型以对齐人类偏好的新兴算法,因其简洁高效而备受关注。相较于传统强化学习方法,DPO 避免了复杂的奖励建模和策略迭代,通过直接比较模型对 chosen 和 rejected 样本的输出进行优化。然而,标准 DPO 采用固定的超参数 $\beta$ 来平衡参考策略和偏好学习,这限制了其在复杂场景下的优化潜力。

固定 $\beta$ 的局限主要体现在两个方面:1)上下文不敏感性,无法适应不同输入上下文的需求;2)优化效率瓶颈,"一刀切"的 $\beta$ 值可能导致在某些情境下学习保守而错失优化机会,或在另一些情境下学习激进而损害已有能力。本研究旨在克服这些局限,提出一种更具泛化能力的 DPO 变体。

\subsection{相关工作}
近年来,大型语言模型的人类偏好对齐研究取得了显著进展。RLHF(Reinforcement Learning from Human Feedback)\cite{ouyang2022training, christiano2017deep}作为一种主流方法,通过人类反馈训练奖励模型,然后使用强化学习优化语言模型的行为。然而,RLHF存在奖励模型训练复杂、强化学习优化不稳定等问题。

DPO \cite{rafailov2023direct} 提出了一种更直接的偏好优化方法,将奖励学习和策略优化融合为一个步骤,显著简化了训练流程。此后,诸多研究者在DPO的基础上提出了各种改进,如 KTO \cite{kassner2023kto}、IPO \cite{azar2023general}等,但这些方法仍然使用固定的 $\beta$ 值来平衡参考策略和偏好学习。

在自适应参数学习方面,相关研究主要集中在学习率自适应\cite{zou2021sufficient, smith2017cyclical}和正则化强度自适应\cite{pan2020adaptive}等方向,但将这一思路应用于DPO算法中的探索还相对缺乏。

\subsection{本文贡献}
本文的主要贡献包括:

\begin{itemize}
    \item 从信息融合的角度重新诠释了DPO算法,将其视为参考策略信息与人类偏好信息的融合过程,提供了理解$\beta$参数作用的新视角。
    \item 提出了Learnable Beta DPO算法,设计了与策略模型紧密耦合的BetaHead网络,能够根据输入上下文自适应地调整$\beta$值。
    \item 基于Qwen-1.5B模型实现了完整的Learnable Beta DPO微调流程,并通过大量实验验证了其有效性。
    \item 通过消融实验深入分析了动态$\beta$计算中各组件的作用,为理解自适应$\beta$的工作机制提供了洞见。
\end{itemize} 

\section{理论基础}
\subsection{标准DPO回顾}
标准DPO的理论基础是Bradley-Terry模型,用于建模成对偏好关系。对于给定上下文$x$和模型输出对$(y_w, y_l)$(winner vs. loser),Bradley-Terry模型假设$y_w$比$y_l$更受偏好的概率为:

\begin{equation}
P(\text{winner} = y_w | x, y_w, y_l) = \frac{\exp(r(x, y_w))}{\exp(r(x, y_w)) + \exp(r(x, y_l))}
\end{equation}

其中$r(x, y)$代表模型输出$y$在上下文$x$下的奖励值。DPO的目标是在不显式学习奖励函数$r(x, y)$的前提下,直接优化策略模型$\pi_\theta(y|x)$。

基于最大似然估计,并假设奖励函数$r(x, y)$与策略模型$\pi_\theta(y|x)$和参考策略$\pi_{\text{ref}}(y|x)$的对数比值成正比:

\begin{equation}
r(x, y) = \beta \log \frac{\pi_\theta(y|x)}{\pi_{\text{ref}}(y|x)}
\end{equation}

将上述奖励函数代入负对数似然损失函数并简化,得到标准DPO Loss函数:

\begin{equation}
\mathcal{L}_{\text{DPO}}(\theta) = - \mathbb{E}_{(x, y_w, y_l) \sim \mathcal{D}} \left[ \log \sigma \left( \beta \left[ \log \frac{\pi_\theta(y_w|x)}{\pi_{\text{ref}}(y_w|x)} - \log \frac{\pi_\theta(y_l|x)}{\pi_{\text{ref}}(y_l|x)} \right] \right) \right]
\end{equation}

其中$\sigma(z) = (1 + e^{-z})^{-1}$为sigmoid函数,$\mathcal{D}$为偏好数据集。

\subsection{固定$\beta$的局限性}
在标准DPO中,$\beta$是一个全局固定的超参数,它控制着模型在参考策略和奖励信息之间的权衡。从优化角度看,$\beta$值的选择至关重要:

\begin{itemize}
    \item 当$\beta$过大时,模型过度依赖参考策略$\pi_{\text{ref}}$,难以充分学习人类偏好
    \item 当$\beta$过小时,模型过度利用奖励信息,可能偏离参考策略太远,导致训练不稳定或出现退化行为
\end{itemize}

固定$\beta$的主要局限在于其无法适应不同输入上下文的需求。在实际应用中,对于模型熟悉的领域(如日常对话),参考策略通常已经具有良好的性能,此时应采用较大的$\beta$值以保持原有能力;而对于棘手的任务(如复杂推理),参考策略的表现可能较差,此时应采用较小的$\beta$值以更多地从人类偏好中学习。固定的$\beta$无法实现这种细粒度的控制,导致整体优化效率受限。

\subsection{信息融合视角下的DPO解析}
我们从信息融合的角度重新解析DPO算法。在DPO中,策略模型$\pi_\theta$需要融合两种信息来源:

\begin{itemize}
    \item 参考策略$\pi_{\text{ref}}$中包含的先验知识(作为信息源1)
    \item 偏好数据集$\mathcal{D}$中包含的人类偏好(作为信息源2)
\end{itemize}

从这一视角看,$\beta$参数实际上控制着两种信息源的相对权重。当两种信息存在冲突时,$\beta$决定了模型更倾向于遵循哪一种信息。通过贝叶斯信息融合理论,我们可以将DPO的优化目标重写为:

\begin{equation}
\pi_\theta^* = \arg\min_{\pi_\theta} \left[ (1-\lambda) \cdot D_{KL}(\pi_\theta || \pi_{\text{ref}}) - \lambda \cdot \mathbb{E}_{(x, y_w, y_l) \sim \mathcal{D}} [\log P(y_w \succ y_l | x, \pi_\theta)] \right]
\end{equation}

其中$\lambda = \frac{1}{1+\beta}$是一个归一化的权重因子。当$\lambda$接近0($\beta$很大)时,模型主要学习参考策略;当$\lambda$接近1($\beta$很小)时,模型主要学习人类偏好。

这种信息融合视角启发我们:理想的$\beta$值应该是与输入上下文$x$相关的函数$\beta(x)$,而非全局固定的常数。 

\section{Learnable Beta DPO方法}
\subsection{BetaHead网络设计}
我们提出的Learnable Beta DPO方法核心在于设计一个与策略模型紧密耦合的BetaHead网络,用于动态计算上下文相关的$\beta$值。BetaHead网络的设计理念是:它应该能够根据模型对输入上下文的熟悉程度来调整$\beta$值,同时保持计算效率和训练稳定性。

BetaHead网络的基本结构如图\ref{fig:beta_head}所示。它接收两个关键输入:

\begin{itemize}
    \item 策略模型计算的上下文困惑度$PPL_{\pi_\theta}(x)$,反映模型对输入的确定性程度
    \item 策略模型最后一层隐状态$h_{\pi_\theta}(x)$,包含输入上下文的深层语义表征
\end{itemize}

\begin{figure}[h]
    \centering
    \includegraphics[width=0.8\linewidth]{figures/beta_head_architecture.pdf}
    \caption{BetaHead网络架构。网络接收上下文困惑度和最后一层隐状态作为输入,输出动态$\beta$值。}
    \label{fig:beta_head}
\end{figure}

\subsection{动态$\beta$计算公式}
我们将动态$\beta$值设计为如下形式:

\begin{equation}
\beta(x) = w \cdot PPL_{\pi_\theta}(x) \cdot f(h_{\pi_\theta}(x))
\end{equation}

其中:
\begin{itemize}
    \item $w$是一个可学习的标量参数,控制$\beta$的整体尺度
    \item $PPL_{\pi_\theta}(x)$是上下文$x$的困惑度,定义为:
    \begin{equation}
    PPL_{\pi_\theta}(x) = \exp \left( - \frac{1}{m} \sum_{i=1}^m \log \pi_\theta(x_i | x_{<i}) \right)
    \end{equation}
    其中$m$是上下文的长度
    \item $f(h_{\pi_\theta}(x))$是一个参数化的修正因子,其取值范围限制在$[1-\epsilon, 1+\epsilon]$,$\epsilon$是一个小的常数(如0.2)。具体实现为:
    \begin{equation}
    f(h_{\pi_\theta}(x)) = 1 + \epsilon \cdot \tanh(NN(h_{\pi_\theta}(x)))
    \end{equation}
    其中$NN(\cdot)$是一个轻量级的多层感知机
\end{itemize}

这种设计有以下优势:
\begin{itemize}
    \item 困惑度$PPL_{\pi_\theta}(x)$提供了一个自动缩放机制:当模型对输入不确定(困惑度高)时,$\beta$值会增大,倾向于保守学习;当模型对输入确定(困惑度低)时,$\beta$值会减小,倾向于激进学习
    \item 修正因子$f(h_{\pi_\theta}(x))$允许模型根据上下文的语义特征微调$\beta$值,提供额外的调节灵活性
    \item 修正因子的取值范围受限,防止$\beta$值变化过大导致训练不稳定
\end{itemize}

\subsection{训练算法}
Learnable Beta DPO的训练算法如算法\ref{alg:learnable_beta_dpo}所示。与标准DPO不同,我们不仅优化策略模型的参数$\theta$,还同时优化BetaHead网络的参数$\phi$(包括标量$w$和修正因子网络$NN$的参数)。

\begin{algorithm}
\caption{Learnable Beta DPO训练算法}
\label{alg:learnable_beta_dpo}
\begin{algorithmic}[1]
\REQUIRE 参考策略 $\pi_{\text{ref}}$, 偏好数据集 $\mathcal{D}$, 初始学习率 $\eta$, BetaHead修正范围 $\epsilon$
\ENSURE 优化后的策略模型 $\pi_\theta$ 和BetaHead网络
\STATE 初始化策略模型参数 $\theta$ (从 $\pi_{\text{ref}}$ 复制)
\STATE 初始化BetaHead网络参数 $\phi$
\WHILE{未收敛}
    \STATE 从 $\mathcal{D}$ 中采样批次 $(x^{(i)}, y_w^{(i)}, y_l^{(i)})_{i=1}^B$
    \FOR{每个样本 $(x^{(i)}, y_w^{(i)}, y_l^{(i)})$}
        \STATE 计算上下文困惑度 $PPL_{\pi_\theta}(x^{(i)})$
        \STATE 提取上下文表征 $h_{\pi_\theta}(x^{(i)})$
        \STATE 通过BetaHead计算动态 $\beta(x^{(i)})$ 值
        \STATE 计算chosen样本的对数概率比 $r_w^{(i)} = \log \frac{\pi_\theta(y_w^{(i)}|x^{(i)})}{\pi_{\text{ref}}(y_w^{(i)}|x^{(i)})}$
        \STATE 计算rejected样本的对数概率比 $r_l^{(i)} = \log \frac{\pi_\theta(y_l^{(i)}|x^{(i)})}{\pi_{\text{ref}}(y_l^{(i)}|x^{(i)})}$
    \ENDFOR
    \STATE 计算批次DPO损失 $\mathcal{L} = -\frac{1}{B} \sum_{i=1}^B \log \sigma(\beta(x^{(i)}) \cdot (r_w^{(i)} - r_l^{(i)}))$
    \STATE 计算梯度 $\nabla_\theta \mathcal{L}$ 和 $\nabla_\phi \mathcal{L}$
    \STATE 更新参数 $\theta \leftarrow \theta - \eta \nabla_\theta \mathcal{L}$
    \STATE 更新参数 $\phi \leftarrow \phi - \eta \nabla_\phi \mathcal{L}$
\ENDWHILE
\end{algorithmic}
\end{algorithm}

需要注意的是,我们采用了端到端的联合优化策略,使BetaHead网络能够与策略模型协同进化。这样设计的好处是:BetaHead网络可以根据策略模型的变化动态调整$\beta$值,而策略模型也能利用更合适的$\beta$值实现更高效的学习。 

\section{实验设置}
% 实验设置章节
这里是实验设置章节内容。将详细介绍模型与数据集、评估指标、对比基线以及具体的实验参数设置。 

\section{实验结果与分析}
% 实验结果章节
这里是实验结果章节内容。将详细分析整体性能比较、不同领域的学习效果分析以及消融实验结果。 

\section{结论与未来工作}
% 结论章节
这里是结论章节内容。将总结本文的主要贡献、局限性以及未来工作方向。 

\bibliographystyle{IEEEtran}
\bibliography{references}

\end{CJK}
\end{document} 