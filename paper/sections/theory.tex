\subsection{标准DPO回顾}
标准DPO的理论基础是Bradley-Terry模型,用于建模成对偏好关系。对于给定上下文$x$和模型输出对$(y_w, y_l)$(winner vs. loser),Bradley-Terry模型假设$y_w$比$y_l$更受偏好的概率为:

\begin{equation}
P(\text{winner} = y_w | x, y_w, y_l) = \frac{\exp(r(x, y_w))}{\exp(r(x, y_w)) + \exp(r(x, y_l))}
\end{equation}

其中$r(x, y)$代表模型输出$y$在上下文$x$下的奖励值。DPO的目标是在不显式学习奖励函数$r(x, y)$的前提下,直接优化策略模型$\pi_\theta(y|x)$。

基于最大似然估计,并假设奖励函数$r(x, y)$与策略模型$\pi_\theta(y|x)$和参考策略$\pi_{\text{ref}}(y|x)$的对数比值成正比:

\begin{equation}
r(x, y) = \beta \log \frac{\pi_\theta(y|x)}{\pi_{\text{ref}}(y|x)}
\end{equation}

将上述奖励函数代入负对数似然损失函数并简化,得到标准DPO Loss函数:

\begin{equation}
\mathcal{L}_{\text{DPO}}(\theta) = - \mathbb{E}_{(x, y_w, y_l) \sim \mathcal{D}} \left[ \log \sigma \left( \beta \left[ \log \frac{\pi_\theta(y_w|x)}{\pi_{\text{ref}}(y_w|x)} - \log \frac{\pi_\theta(y_l|x)}{\pi_{\text{ref}}(y_l|x)} \right] \right) \right]
\end{equation}

其中$\sigma(z) = (1 + e^{-z})^{-1}$为sigmoid函数,$\mathcal{D}$为偏好数据集。

\subsection{固定$\beta$的局限性}
在标准DPO中,$\beta$是一个全局固定的超参数,它控制着模型在参考策略和奖励信息之间的权衡。从优化角度看,$\beta$值的选择至关重要:

\begin{itemize}
    \item 当$\beta$过大时,模型过度依赖参考策略$\pi_{\text{ref}}$,难以充分学习人类偏好
    \item 当$\beta$过小时,模型过度利用奖励信息,可能偏离参考策略太远,导致训练不稳定或出现退化行为
\end{itemize}

固定$\beta$的主要局限在于其无法适应不同输入上下文的需求。在实际应用中,对于模型熟悉的领域(如日常对话),参考策略通常已经具有良好的性能,此时应采用较大的$\beta$值以保持原有能力;而对于棘手的任务(如复杂推理),参考策略的表现可能较差,此时应采用较小的$\beta$值以更多地从人类偏好中学习。固定的$\beta$无法实现这种细粒度的控制,导致整体优化效率受限。

\subsection{信息融合视角下的DPO解析}
我们从信息融合的角度重新解析DPO算法。在DPO中,策略模型$\pi_\theta$需要融合两种信息来源:

\begin{itemize}
    \item 参考策略$\pi_{\text{ref}}$中包含的先验知识(作为信息源1)
    \item 偏好数据集$\mathcal{D}$中包含的人类偏好(作为信息源2)
\end{itemize}

从这一视角看,$\beta$参数实际上控制着两种信息源的相对权重。当两种信息存在冲突时,$\beta$决定了模型更倾向于遵循哪一种信息。通过贝叶斯信息融合理论,我们可以将DPO的优化目标重写为:

\begin{equation}
\pi_\theta^* = \arg\min_{\pi_\theta} \left[ (1-\lambda) \cdot D_{KL}(\pi_\theta || \pi_{\text{ref}}) - \lambda \cdot \mathbb{E}_{(x, y_w, y_l) \sim \mathcal{D}} [\log P(y_w \succ y_l | x, \pi_\theta)] \right]
\end{equation}

其中$\lambda = \frac{1}{1+\beta}$是一个归一化的权重因子。当$\lambda$接近0($\beta$很大)时,模型主要学习参考策略;当$\lambda$接近1($\beta$很小)时,模型主要学习人类偏好。

这种信息融合视角启发我们:理想的$\beta$值应该是与输入上下文$x$相关的函数$\beta(x)$,而非全局固定的常数。 