\subsection{模型与数据集}
\subsubsection{模型选择}
我们以Qwen-1.5B作为基础模型进行实验,具体实现上使用了DeepSeek-R1-Distill-Qwen-1.5B,这是一个基于Qwen的高效蒸馏版本。选择该模型的原因包括:1)参数量适中(1.5B),计算资源需求合理;2)在中英文任务上均有良好表现;3)开源许可允许研究使用。

为了实现Learnable Beta DPO,我们在模型基础上添加了BetaHead网络。BetaHead网络的具体配置如下:
\begin{itemize}
    \item 一个可学习的标量参数$w$,初始化为0.1
    \item 一个双层MLP,输入维度为模型隐状态维度(2048),中间层维度为512,输出维度为1
    \item 修正因子范围参数$\epsilon$设置为0.2
\end{itemize}

\subsubsection{数据集}
我们使用以下数据集进行实验:

\begin{itemize}
    \item \textbf{Anthropic Helpful and Harmless (HH)} \cite{bai2022training}:包含52k条助手回复偏好对,覆盖多种通用对话场景
    \item \textbf{Stanford Human Preferences (SHP)} \cite{pmlr-v162-ethayarajh22a}:包含385k条人类偏好数据,侧重于回复的有用性、诚实性和无害性
    \item \textbf{UltraFeedback} \cite{cui2023ultrafeedback}:包含64k条高质量反馈数据,每个样本有人类评分和详细反馈
\end{itemize}

我们按照8:1:1的比例将每个数据集分为训练集、验证集和测试集。

\subsection{评估指标}
为全面评估Learnable Beta DPO的性能,我们采用以下评估指标:

\begin{itemize}
    \item \textbf{偏好一致率(Preference Agreement)}:模型输出与人类偏好的一致程度,即模型给出的chosen响应概率高于rejected响应的样本比例
    \item \textbf{KL散度(KL Divergence)}:微调后模型与参考模型的KL散度,用于衡量模型偏离参考策略的程度
    \item \textbf{赢率(Win Rate)}:由人类评估者判断,微调后模型相比基线模型的胜率
    \item \textbf{分领域性能(Domain-specific Performance)}:在不同领域(日常对话、创意写作、逻辑推理等)的表现
    \item \textbf{$\beta$分布分析}:记录不同上下文下$\beta$值的分布情况,分析其与任务难度的相关性
\end{itemize}

自动评估使用验证集和测试集进行,人工评估从测试集中随机采样200个样本,由5名评估者独立判断。

\subsection{对比基线}
我们将Learnable Beta DPO与以下基线方法进行对比:

\begin{itemize}
    \item \textbf{原始SFT模型}:仅通过监督微调的模型,作为基础参考
    \item \textbf{标准DPO}:使用固定$\beta$值的DPO,我们测试了三种$\beta$值设置:0.1(较小)、1(中等)和10(较大)
    \item \textbf{IPO} \cite{azar2023general}:Implicit Preference Optimization,一种改进的DPO变体
    \item \textbf{KTO} \cite{kassner2023kto}:KL-regularized Preference Optimization,另一种DPO变体
\end{itemize}

\subsection{实验设置}
我们的实验配置如下:

\begin{itemize}
    \item \textbf{训练设置}:批次大小为64,使用Adam优化器,学习率为5e-6,使用余弦学习率调度,训练3个epochs
    \item \textbf{硬件}:4×NVIDIA A100 GPU (40GB)
    \item \textbf{参数冻结}:仅微调模型最后4层和BetaHead网络,其余参数保持冻结
    \item \textbf{超参数}:使用学习率为[1e-6, 3e-6, 5e-6, 1e-5]和修正因子范围$\epsilon$为[0.1, 0.2, 0.3]进行超参数搜索
\end{itemize}

所有实验均重复运行3次,报告平均性能和标准差。为确保实验的公平性,所有基线方法都使用相同的训练数据、批次大小和总训练步数。 